Die vorliegende Ausarbeitung dokumentiert die  Erstellung einer Hyperledger-Fabric Applikation zum Management eines Marktes für Energie im Rahmen des Kurses "Blockchain im Internet of Things" im Wintersemester 20/21 an der "Frankfurt University Of Applied Sciences". Zum Zwecke der Übersichtlichkeit Gliedert sich das vorliegende Dokument in Sechs Kapitel. "Einleitung" gibt einen Überblick über unsere Arbeitsweise, Team-Hierarchie sowie unser Szenario. im Kapitel "Foundation" werfen wir einen Blick auf den Aufbau und die Möglichkeiten welche Hyperledger Fabric bietet , während wir bereits zu differenzieren versuchen welche davon im Rahmen des "System Designs" für das von uns gewählte Szenario relevant sind. In welchem wir dann auch unsere Gründe für die von uns getroffenen Designentscheidungen erläutern.Dazu präsentieren wir einige der von uns erstellten Diagramme und zeigen deren nutzen als Leitbild bei der Finalisierung unseres Projektes. "System Implementation" befasst sich dann schließlich mit der technischen Details unserer Applikation während wir hier ebenfalls dem Leser diverse Funktionsweisen anhand von Codefragementen nahe bringen gehen wir darüber hinaus auf einige Problematiken ein welche uns spezielle Schwierigkeiten bei der Implementierung machten und wie wir diese lösten. "Validation" (DAS FÜGT HENRY EIN SOBALD ER WEISS WAS ZUM FICK IN DIESEM KAPITEL STEHEN SOLL). In "Fazit" reflektieren wir schließlich die Gesamtheit des Projekts. Wir kommentieren unsere Arbeit als Gruppe und unsere Meinung von Hyperledger Fabric und unserem Szenario. Darüber hinaus geben wir einen Ausblick auf Features welche wir mit mehr Zeit gerne implementiert hätten und Probleme die wir gerne noch beheben würden.

\section{Organisation}

Die Teamarbeit erfolgte in einer sehr flachen Hierarchie. Uns war im Verlauf der Ausarbeitung sehr viel daran gelegen dass alle sich einig über die zu erfolgenden Schritte  waren. Um dies zu erreichen planten wir drei regelmäßige Treffen pro Wochen in welchen wir unser Vorgehen bis zum jeweils nächsten planten , immer mit dem erklärten Ziel die von uns als Sprint-Ziele deklarierten Ziele zu erreichen. Bei diesen Treffen war es irrelevant wie viel Fortschritt von den jeweiligen Team-Mitgliedern zu diesem Zeitpunkt erarbeitet wurde, es ging hauptsächlich um den regelmäßigen Kontakt und das Einvernehmen über das weitere vorgehen.a Zu jedem dieser Treffen wurde darüber hinaus ein Protokoll gefertigt welches jedem Mitglied zur Verfügung stand um ggf. Verhinderten die Möglichkeit zu eröffnen sich trotzdem den neusten Stand zu bringen. Darüber hinaus versuchten wenn möglich klar Aufgabenzuteilung zu vermeiden , wir ordneten gewissen Features stattdessen eine Priorität zu und ließen diejenigen daran Arbeiten welche gerade die größte Motivation auf die jeweilige Task hatten. Anschließend kombinierten wir die Ergebnisse  als Gruppe. Der Hintergedanke hierbei war jener das Menschen natürlicherweise die besten Ergebnisse Liefern wenn Sie an Dingen arbeiten an welchen Sie Spaß haben.

\section{Tools}
Zur Organisation, Kommunikation, Code´-Verarbeitung und Versionskontrolle.

\begin{enumerate}
\item WhatsApp
\item Discord
\item Git-Hub
\item Zoom
\item Google Drive
\end{enumerate}
WhatsApp diente zur regelmäßigen und formlosen Absprache untereinander, zum Austausch von Ideen und um andere über akute Probleme zu informieren oder kurzfristige Terminabsprachen zu treffen. Auf Discord hingegen betrieben wir einen Server, hauptsächlich um konzentriert Links zu nützlichen quellen zu bündeln und sekundär um einen weiteren Kommunikationskanäle gewährleisten zu können. Darüber hinaus verwendeten wir jedoch Zoom für regelmäßige Videocall-Meetings. Google Drive, also Cloudstorage-Anbieter, diente uns zur Sammlung sämtlicher nicht Quellcode oder Netzwerk bezogener Daten, hier finden sich in einem Geteilten Ordner die Sammlung unsere Sprintreviews, Präsentation und erstellter Diagramme. Unseren Quellcode schließlich managten wir über das Versionmanagementtool Git-Hub in welchem auch unsere offenen Tasks verfolgt wurden. Die Wahl der verwendeten IDE blieb schließlich jedem selbst überlassen.
