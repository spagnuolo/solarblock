Im Rahmen des folgenden Kapitels werden wir den Kontext darlegen in welchem unsere Applikation entwickelt wurde. Im Vorfeld des ersten Sprints hielten wir ein Problemstellung fest welche wir mit Hilfe einer Blockchain unter Hyperledger Fabric zu lösten. Wir erläutern nunmehr die Abgrenzungen des Szenarios, legen in abstrahierter Form die Probleme da die es zu lösen galt und unsere  Begründung wieso wir dieses Szenario als Beispiel ausgewählt haben.  Einen ex post wertenden Blick auf jenes Rationalem liefern wir ergänzend in Kapitel 5.

\section{Abgrenzung des Szenarios}
Das Kernkonzept welchem sich unser Szenario widmet ist die Solarenergie. Wir haben uns zum Ziel gesetzte einen privaten, angebotsbasierten Markt für Solarenergie zu schaffen. Dieser soll im wesentlichen den Teilnehmern ermöglichen im Rahmen des Privathaushaltes via Solarpaneelen generierten Strom Mitbürgern entgeltlich zur Verfügung zu stellen, bzw. ihren Bedarf aus der Solarenergie ihrer Mitbürger zu decken. Die Grundlage für einen solchen Austausch stellt das lokale Stromnetz da. Dies bindet den jeweiligen Netzbetreiber ein welcher als Broker agiert der ebenfalls Solarenergie ankaufen kann wenn sich dafür sonst kein privater Abnehmer findet, jedoch selbst keine eigenen Angebote ins Netzwerk emittiert.

\subsection{Teilnehmer und deren Motivation}
Hiermit lassen sich also zwei Teilnehmergruppen identifizieren welche verschiedene Interessen besitzen. Für beide lässt sich als Motivation entsprechen der grundlegende Hang des Homo oeconomicus\footnote{Homo oeconomicus beschreibt in den Wirtschaftswissenschaften das Axiom eines Akteurs welcher im ökonomsichen Gesamtkonstrukt steets nutzenmaximiert handelt} zur Profitmaximierung festhalten, auch wenn diese auf unterschiedlichen wegen erreicht wird. Die privaten Haushalte welche in das System einsteigen erhalten die Möglichkeit nicht genutzter Energie in Kapital umzusetzen, dies geschieht durch die entgeltliche zur Nutzungstellung dieser Energie gegenüber der anderen Teilnehmer.  Mit der weiteren Intention dass andere Teilnehmer auf diese weise günstiger als auf dem herkömmlicher Wege an eben selbige Energie kommen. Die Motivation welche den Netzbetreiber in diesen Konstrukt treibt ist zwar in ihrer Natur auch Profitorientiert geartet, erreicht die Profitoptimierung jedoch eher auf sekundäre Art und weise.  Ein solches Netzwerk bietet zwei Vorteile für den Netzbetreiber indem es zum einen die Möglichkeit bietet geringere Defizite in der Energieversorgung durch den Ankauf von Strom aus dem privaten Netz zu decken. Zum anderen erreichen wir auf diese Art eine Entlastung der Verteilernetze da der Strom in der Tendenz eher lokal bleibt. Wenn wir unsere Aufmerksamkeit jedoch einmal von den Teilnehmern weg lenken können wir darüber hin aus jedoch noch einige Vorteile erwägen welche als gesamtgesellschaftliches Interesse klassifizierbar sind. So begünstigen und fördern wir mit einem solchen  Konstrukt die Unabhängigkeit der Teilnehmenden Haushalte von etablierten Netzbetreibern und dadurch die Autarkie dieses Haushalte. Im weiteren würde eine großflächige Entwicklung zu einer solchen solarbasierten Technik umwelttechnisch langfristig positive Auswirkungen besitzen.

\section{Probleme}
Das hier kommt zur Validation!
